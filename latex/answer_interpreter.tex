This component receives the answers of the Question Answering System, interprets them and outputs the \textit{tokens for predictions}.

\subsection{Stemming}
\label{sec:stemming}
Before the interpretation, the answers are stemmed using the \textit{Snowball Stemmer} present in NLTK with a slight modification \cite{snowballstemmer, nltk}. This because it is true that stemming reduces the variability of a text, mapping inflected or derived words into their root form (the stem); but it is also true that a stem does not need to be identical to the morphological word root. This is a problem because a stemmer might output out-of-vocabulary words, which cannot be vectorified. Thus, the solution stands in accepting the stemmed word only if it is inside the vocabulary (the list of the first $100\,000$ terms in GloVe pre-trained embeddings). This stemmer is also used in the calculation of the representative embeddings of symptoms.

\subsection{Filtering ``useless words''}
The stemmed answers are then passed to a function that filters ``useless words''. Basically, it keeps only words that are tagged as a \textit{part of speech} that is inside the following list: noun, adjective, coordinating conjunction, punctuation, verb and auxiliar. For this purpose the Stanford POS Tagger, which is inside Stanford NLP library \cite{stanfordnlp}, was used. The basic idea is to simplify the answer deleting words like adverbs and pronouns that do not contribute to the meaning of the symptoms.

\subsection{Answer interpretation}
The two different types of answers are processed in a different way:
\begin{itemize}
  \item the general answer about symptoms is one for each passage. Usually, this answer contains more symptoms separated by a coordinating conjunction: mainly, the comma and ``and''. For this reason, the Answer Interpreter splits the answer on these terms and originates one or more \textit{tokens for predictions}.
  \item the specific answers about a body part are as many as the number of body parts found in the passage. Given a body-part answer, the \textit{token for prediction} is formed concatenating the answer with the name of the body part.
\end{itemize}

\begin{figure}[h]
\centering
\includegraphics[width=13cm]{answer_interpreter}
\caption{The Answer Interpreter illustrated.}
\medskip
\label{fig:answer_int}
\end{figure}
