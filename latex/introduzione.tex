\chapter{Introduction}
\label{cha:intro}

\section{The major problem of physicians: time}
\label{sec:problem_doctors}

Physicians all over the world visit patients under time limitations. In addition, in underdeveloped countries there is not always a suitable doctor-population ratio (1:1000 is the optimal ratio according to the World Health Organization): thus, these doctors may be overloaded with work. Hence, the main problem of physicians is time. This obviously affects the way in which doctors examine patients, especially the anamnesis, which is the most time-consuming task during medical examinations.

Anamnesis, often referred in medical jargon as “history taking”, is a long task because the physician does not have only to interview the patient and gather information about the history of their problem, but even write down a brief summary of the conversation in the patient’s record. This necessary bureaucracy is demanding and risks becoming all-embracing. To give an example, a study on 57 physicians in the United States observed that during the clinic day, physicians spend 49.2\% of their clinical time on health records and desk work and only 27\% of their time with patients (Sinsky, 2016). However, this task is fundamental: according to research, medical history provides in the 82.5\% of patients enough information to make an initial diagnosis of a disease (Hampton, 1975).

The traditional way of interviewing the patient is not only long and demanding, but even sometimes incomplete. For instance, 134 primary care physicians observed that only 59\% of essential history items were collected during history taking. The physicians were able to obtain relevant information about presenting symptoms and medications, but they often missed important information about related symptoms and medical history. This because a physician needs to remember a large number of questions and symptoms to have a complete and accurate history. Human memory is fallible and forgetting to ask some relevant questions during an interview might have a significant implication in patient management. (John document)

Therefore, it is for these reasons that a new approach to anamnesis should be tried. Digitize this task is a possible solution to the aforementioned problems which is time-saving for physicians and, at the same time, could provide a more effective history taking outcome. The ideal interaction model between this history taking program and the patient is a chat conversation. Basically, the patient, chatting, interacts with this conversational software, which has to understand and classify the presenting symptoms and, later, elicit symptoms and information useful for the diagnosis that the patient may have forgotten. As the title of my dissertation suggests, I focused on the first obstacle towards an effective history taking chatbot: identifying and classifying symptoms within a patient’s answer.

\section{}
\label{sec:problem_doctors}

