\chapter{Methods}
\label{cha:methods}
In this chapter I will describe in detail the methods used in order to achieve the results described in the next chapter. In the following sections I will analyze the abstract components illustrated in the Introduction.

\section{The \textit{Body Part Finder}}
\label{sec:body_part_finder}
\subsection{The body part list}
The objective of this component is to identify body parts words within the patient's sentence. For doing so, it is obviously necessary to have a list of body parts. In addition, each body part can be expressed by the patient using a large variety of terms (its synonyms). Hence, with the addition of an abstraction layer, the list of body parts becomes a list of body parts concepts; and each body part concept is associated with a list of its synonyms. Another problem to face is that the language of this body parts classification should be at “patient level”: therefore it has to be informal and not strictly medical.

Unfortunately, I did not find any body part classification with these features. Therefore, the only chance for me was to produce my own classification, starting my work from the numerous informal body part lists that I found searching the Web. I based my classification mainly on the Wikipedia's list of body parts[], which I completed with other missing body parts found in other lists. I understand the licit perplexities that this approach may arise. However, in my opinion, what really matters is that, even with this amateur list of body parts, the results make my proof of concept creditable, as we will see in the next chapter. The classification is composed of [130] body parts saved in a csv file.

\subsection{Finding a body part in a sentence}
Finding a body part in a sentence is quite easy if you have an exhaustive list of body parts and the respective synonyms. The code does nothing more than a string matching between each word in the sentence and each synonym present inside the classification. This approach is simple but not perfect: for example, the word “back” is both a preposition and a body part. Therefore, looking only at words means forgetting the context: this may introduce misunderstandings and lower the quality of the results.
%% TODO scrivo che c'e' da migliorare

\section{Question-Answering system}
\label{sec:qa_system}
The outcome of the \textit{Body Part Finder} is then passed to the \textit{QA system}, whose basic aim is to give an answer to questions like “What are the symptoms in the sentence?” or “What is the problem with the neck?”. If the reader has never heard of Machine Reading Comprehension, this may seem magic. However, fortunately, as we will see in this section, it is not.

\subsection{Types of questions}
The questions asked to the specified model over each sentence are of two types:
\begin{enumerate}
  \item generic question about the symptoms in the sentence (“What are the symptoms?”)
  \item specific question about the problem of a found body part in the sentence (“What is the problem with the [body part]?”)
\end{enumerate}

The question over the symptoms is intentionally generic because its aim is to catch the majority of the symptoms within the sentence. Thus, as we will see in the next section, the answer to this question should be interpreted.

At the same time, the question over a body part is deliberately specific. The logic behind this question is that if a patient refers to a body part during the interview, perhaps it is the subject of a problem.

Here is a figure in order to understand the whole component:

%% IMAGE ABOUT QA SYSTEM
\begin{figure}[h]
\centering
\includegraphics[width=14cm]{qa_sys}
\caption{The \textit{QA system} illustated.}
\medskip
\end{figure}

If we look at the answers of this figure, we can understand two problems: the model can sometimes provide an useful answer but too generic (this is the case of ``small bumps'') and might provide redundant answers, increasing unnecessarily the number of predictions.

\subsection{BERT}
In this subsection the main characteristics of BERT will be outlined.

BERT, which stands for Bidirectional Encoder Representations from Transformers, is a new language representation model. Even if BERT was presented to the research world in May 2019, it has yet obtained new state-of-the-art results on eleven natural language processing tasks. For instance, a finetuned BERT push SQuAD v1.1 question answering F1 Test to 93.2 (1.5 points of absolute improvement) and SQuAD v2.0 F1 Test to 83.1 (5.1 points of absolute improvement) \cite{bert}.

The training framework proposed by the authors of BERT is composed of two steps:
\begin{enumerate}
  \item \textit{pre-training}: BERT is trained over different pre-training tasks.
  \item \textit{fine-tuning} over a single downstream task: in this case, the SQuAD task.
\end{enumerate}

BERT’s  model  architecture is a multi-layer bidirectional Transformer encoder. These last components, that are the building blocks of BERT, are implemented as described in \cite{attentionisallyouneed}.

Considering BERT Base, the size taken in consideration for this project, the hidden layers (i.e. Transformers encoders) are $12$, the hidden size is $768$ and the number of self-attention heads is $12$.

\begin{figure}[t]
\centering
\includegraphics[width=18cm]{bert-for-squad-original-paper}
\caption{Overall pre-training and fine-tuning procedures for BERT. The pre-training procedure provides a basis for the numerous possible downstream tasks. \texttt{[CLS]}, which stands for ``classification'', is a special symbol added in front of every input example, and \texttt{[SEP]} is a special separator token.\footnote{For further information about the input representation, see \cite{bert}}Image taken from \cite{bert}.}
\medskip
\label{fig:bertsquad}
\end{figure}

\subsubsection{Pre-training BERT}
Briefly, pre-training BERT means training it on two tasks:
\begin{itemize}
  \item \textit{Masked Language Model} (MLM), often called \textit{Cloze task}. A bidirectional model like BERT is undoubtedly more powerful than a left-context model like the OpenAI GPT Transformer. But this type of model is even more difficult to train because a bidirectional conditional language model cannot be trained left-to-right or right-to-left, since this would allow each word to indirectly “see itself” during training.

  Thus, in order to train a bidirectional representation, the researchers simply mask some percentage of the input tokens at random (15\% in their experiments), and then predict those masked tokens. This means substitute a real token with a placeholder token \texttt{[MASK]}.
  
  \item \textit{Next Sentence Prediction} (NSP). This task is beneficial to many important downstream tasks like Question Answering and Natural Language Inference that are based on understanding the relation between two sentences, skill that is not captured by the previous task. Basically, it is a binary \textit{next sentence prediction}: when choosing the sentences \texttt{A} and \texttt{B} for the training, in the 50\% of the cases \texttt{B} is an actual next sentence of \texttt{A} (and it is labeled as \texttt{IsNext}) while in the other cases the choice of \texttt{B} is random between the sentences (labeled as \texttt{NotNext}).
\end{itemize}

\subsubsection{A dataset for Reading Comprehension: SQuAD}
\label{squad}
In the fine-tuning stage BERT is trained on SQuAD, a dataset for Reading Comprehension developed by Stanford University \cite{squad}. It consists in a collection of $100\,000$ question-answer pairs over different Wikipedia passages. There are two versions of SQuAD: the first version (\texttt{v1.1}) contains only questions over a passage that have an answer, while the second version (\texttt{v2.0}) has some questions without answer. Obviously, doing better on SQuAD 2.0 is more difficult, because it requires a minimum level of reasoning.

\subsubsection{Fine-tuning BERT on SQuAD1.1}
As illustrated in Figure \ref{fig:bertsquad}, during fine-tuning passage and question are both passed to BERT (separated by the \texttt{[SEP]} tag). This phase introduces two trainable vectors: $S$ (for the start of the answer) and $E$ (for the end of the answer) both of dimension $\mathbb{R}^\mathbb{H}$ ($\mathbb{H}$ is the hidden size, $768$ for BERT Base). 

The probability of a token $i$ of being the start of the answer is:
\begin{equation}
P_{i} = \frac{\mathrm{e}^{S \cdot T_{i}}}{\sum_{j} \mathrm{e}^{S \cdot T_{j}}}
\end{equation}
where $T$ is the last layer of BERT. The analogous formula is used for the end of the answer.

The score of a candidate span that goes from token $i$ to token $j$ is defined as:
\begin{equation}
\text{score}_{i, j} = S \cdot T_{i} + E \cdot T_{j}
\end{equation}

The maximum score is reached where $j \geq i$ is used as prediction.

Finally, the training objective is defined as the sum of the two log-likelihoods \eqref{eqn:loglikelihood} of the correct start and end positions.

\begin{equation}
\label{eqn:loglikelihood}
\text{loglikelihood}(S) = \sum_{i} y_{i} \mathrm{ln} (P(y_{i} | S)) + (1 - y_{i}) \mathrm{ln} (1 - P(y_{i} | S))
\end{equation}

For the purpose of this work, the pretrained BERT-SQuAD model by DeepPavlov \cite{deeppavlov} was used.


\subsection{R-NET}
I used Deepavlov R-NET trained on SQuAD



\section{Answer Interpreter}
\label{sec:answer_interpreter}
This component receives the answers of the Question-Answering system, interprets them and outputs \textit{tokens for predictions}.
% stemming
\subsection{Stemming}
\label{sec:stemming}
Before the interpretation, the answers are stemmed using the \textit{Snowball Stemmer} present in NLTK with a slight modification. This because it is true that stemming reduces the variability of text, mapping inflected or derived words into their root form (the stem); but it is also true that a stem need not be identical to the morphological word root. This is a problem because a stemmer might output out-of-vocabulary words, which cannot be vectorified. Thus, the solution to this problem stands in accepting the stemmed word only if it is inside the vocabulary (the list of the first $100\,000$ terms in GloVe pre-trained embeddings). This stemmer is also used in the calculation of the representative embeddings of symptoms (which will be discussed later).

\subsection{Deleting ``unuseful words''}
The stemmed answers are then passed to a function that deletes ``unuseful words''. Basically, this function keeps only words that are tagged as a \textit{part of speech} that is inside the following list: noun, adjective, coordinating conjunction, punctuation, verb and auxiliar. For this purpose I used the Stanford POS Tagger, which is inside Stanford NLP library. Thus, the basic idea is to simplify the answer deleting words like adverbs and pronouns that do not contribute to the meaning of the symptoms.

\subsection{Answer interpretation}
The two different types of answers are processed in a different way:
\begin{itemize}
  \item the general answer about symptoms is one for each passage. Usually, this answer contains more symptoms separated by a coordinating conjunction (mainly, comma and ``and''). For this reason, the Answer Interpreter splits the answer on these particles and originates one or more \textit{tokens for predictions}.
  \item the specific answers about a body part are as many as the number of body parts found in the passage. Given a body-part answer, the \textit{token for prediction} is formed concatenating the answer with the name of the body part.
\end{itemize}

\begin{figure}[h]
\centering
\includegraphics[width=13cm]{answer_interpreter}
\caption{The Answer Interpreter illustrated}
\medskip
\label{fig:answer_int}
\end{figure}



\section{Vectorifier}
\label{sec:vectorifier}
Essentially, this component encodes a set of words in a vector. The supported internal representation are two:
\begin{itemize}
  % default pooling settings of bert as a service: mean of the second to last layer
  \item \textit{BERT embeddings}. Using BERT is even possible to extract embeddings, which have the virtue of being contextual. For this purpose I used ``bert-as-a-service''. It works with different pooling strategies: with the default one, does a mean of the vectors of second-to-last layer.
  \item \textit{GloVe embeddings}.
\end{itemize}



\section{Symptom Tree Mapper}
\label{sec:symptom_tree_mapper}
The aim of this component is taking the \textit{tokens for predictions} (outcome of Answer Interpreter) and with the help of the Vectorifier, output the most similar symptom's CUI.

%scrivo dello stemming
In order to do this, each symptom inside the classification must have its representative vector of the same type and dimension of the embedding of the \textit{token for prediction}. Since each symptom in the classification has a descriptive field, called ``concept name'', I treated it like a sentence and I choose to use its vectorization as representative vector. In order to hasten the whole process at runtime I pre-computed these representative vectors for each supported type and dimension.

\subsection{The Symptom Tree}
As mentioned in section \ref{sec:cla_symp}, a suitable data structure for representing the symptom classification is a tree. In the UMLS system this hierarchy is stored with a parent array representation. The Python library that I used for this purpose is ``anytree''.

During the construction of the tree I tagged some nodes as related with a specific body part. This is helpful during mapping: given a \textit{token for prediction} that contains a body part, the search of the correct symptom is then circumscribed only to some subtrees.

\subsection{Mapping a \textit{token for prediction} to a symptom node}



\section{How to evaluate results?}
\label{sec:eval_results}
TODO


\section{Dataset}
site I visited
"extrapolating sentences within the posts" in which there were classifiable symptoms wrt the classification
format of the sentences: xml tags, sentence tag