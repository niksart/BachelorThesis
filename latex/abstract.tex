\chapter*{Abstract} % senza numerazione
\label{abstract}

\addcontentsline{toc}{chapter}{Abstract} % da aggiungere comunque all'indice

%%%%%%%%%%%%%%%%%%%%%%%%%%%%%%%%%%%%%%%%%%%%%%%%%%%%%%%%%%%%%%%%%%%%%%%%%%%%%%%%
%% Sommario è un breve riassunto del lavoro svolto dove si descrive l'obiettivo, l'oggetto della tesi, le 
%% metodologie e le tecniche usate, i dati elaborati e la spiegazione delle conclusioni alle quali siete arrivati.  
%%
%% Il sommario dell’elaborato consiste al massimo di 3 pagine e deve contenere le seguenti informazioni:
%%  - contesto e motivazioni 
%%  - breve riassunto del problema affrontato
%%  - tecniche utilizzate e/o sviluppate
%%  - risultati raggiunti, sottolineando il contributo personale del laureando
%%
%% Summary:
%%
%%  - briefly and abstractly explain the problem and what my code does (example from dataset, image)
%%    - 2 objectives:
%%      - identify for each symptom a part of sentence related to it (this are intelligible tokens and then can be printed in the final report)
%%      - mapping these parts of sentences in a specific symptom identified by a CUI
%%    - my work is part of a bigger project: an history taking chatbot (image)
%%      - what is the aim of the history taking chatbot?
%%      - why is it useful to map the symptoms towards a classification of symptoms?
%%        - briefly tell what is MEDCIN and UMLS classification
%%
%%  - why history taking? => document john
%%    - why is it necessary? What does a physician lack? => document john
%%      - lack of physicians on rural areas / less populated areas (Philippines example)
%%      
%%  - abstract but more detailed paragraph about code and techniques
%%    - used techniques
%%    - developed techniques
%%    
%%  - results
%%    - QA system results
%%    - mapping towards a classification results
%%%%%%%%%%%%%%%%%%%%%%%%%%%%%%%%%%%%%%%%%%%%%%%%%%%%%%%%%%%%%%%%%%%%%%%%%%%%%%%%
