\chapter{Results}
\label{cha:results}
In the next two sections I will illustrate and discuss the results of the main two abstract components of my project: the \textit{Symptom Identifier} and the \textit{Symptom Classifier}.

\section{Evaluation of the Symptom Identifier}
In order to evaluate the Symptom Identifier, I manually classified the tokens for predictions of each sentence in 4 classes (just to recall, tokens for predictions are the outcome of the Symptom Identifier):
\begin{itemize}
  \item \texttt{correct}. These are tokens that contain an intelligible symptom that is also present in the patient's sentence;
  \item \texttt{redundant}. These tokens are correct but redundant (there is another token that has the same meaning);
  \item \texttt{non sense}. These tokens are simply not intelligible;
  \item \texttt{wrong}. These tokens are deceptive because they contain an intelligible symptom but they are not present in patient's sentence.
\end{itemize}

Finally, for each sentence, I noted also the number of symptoms present in the sentence but not in the tokens (\texttt{missing}). In order to choose the best model for future evaluations, I designed a score metric that summarizes the performance of this component:
\begin{equation}
score = \texttt{\#correct} + 0 \cdot \texttt{\#redundant} - \texttt{\#non sense} - \texttt{\#wrong} - \texttt{\#missing}
\end{equation}

The reader may ask their-self why is a non sense token so penalized. The reason is that this type of token is detrimental as much as a wrong one because it leads to a random prediction. Instead, a redundant token is neutral because it probably leads to a duplicated prediction.

Here is an overview of the obtained results:

\definecolor{green}{rgb}{0.8,0.9,0.8}
\definecolor{lightgreen}{rgb}{0.89, 0.94, 0.89}
\definecolor{red}{rgb}{1,0.89,0.89}

\begin{center}
 \begin{tabular}{| c | c | c | c |} 
 \hline
 \# of & BERT & R-NET \\ [0.5ex] 
 \hline\hline
 \rowcolor{green}
 \texttt{correct} & 423 & 427 \\ 
 \hline
 \rowcolor{lightgreen}
 \texttt{redundant} & 275 & 195 \\
 \hline
 \rowcolor{red}
 \texttt{non sense} & 122 & 247 \\
 \hline
 \rowcolor{red}
 \texttt{wrong} & 8 & 14 \\
 \hline
 \rowcolor{red}
 \texttt{missing} & 61 & 74 \\
 \hline
  \textbf{score} & \textbf{232} & \textbf{92} \\ 
 \hline
\end{tabular}
\end{center}


\section{Evaluation of the Symptom Classifier}
\subsection{Definitions and notation}

The expression ``\texttt{real cuis}'' is always contextual to a specific sentence. Thus, ``$\texttt{real cuis}_{i}$'' indicates the list of real symptom CUIs within the i-th sentence. Analogously, the expression ``$\texttt{predicted cuis}_{i}$'' indicates the list of predicted CUIs of the i-th sentence.

The symbol ``\texttt{\#}'' in front of a list identifier indicates its length.

\subsubsection{Types of predictions}
The predictions extracted from a patient's sentence can be classified:
\begin{itemize}
  \item \texttt{correct}, if the prediction is located into a subtree rooted in any of the \texttt{real cuis}. Correct predictions can be of two types:
    \begin{itemize}
      \item \textit{redundant} (\texttt{R}). When a prediction is mapped into a subtree, the subtree is marked as associated with that prediction. Then, if another prediction of the same sentence stays in that subtree, it is classified as redundant;
      \item \textit{non-redundant} (\texttt{NR}), if the prediction stays in a unmarked subtree;
    \end{itemize}
  \item \texttt{wrong}, if the prediction is not located in any of the subtrees.
\end{itemize}

Also these classifications are contextual to the sentence, thus they can be indexed like in the case of \texttt{real cuis}.

\subsection{Measures}
%TODO

Accuracy
\begin{equation}
\frac{\sum_{i}{\texttt{\#correct}_{i}^{\texttt{NR}}}}{\sum_{i}{\texttt{\#real cuis}_{i}}}
\end{equation}

correct \% of predictions
\begin{equation}
\frac{\sum_{i}{\texttt{\#correct}_{i}}}{\sum_{i}{\texttt{\#predicted cuis}_{i}}}
\end{equation}

attempts per real cui
\begin{equation}
\frac{\sum_{i}{\texttt{\#predicted cuis}_{i}}}{\sum_{i}{\texttt{\#real cuis}_{i}}}
\end{equation}

missed real cuis
\begin{equation}
\sum_{i}{\texttt{\#real cuis}_{i}} - \sum_{i}{\texttt{\#correct}^{\texttt{NR}}_{i}}
\end{equation}