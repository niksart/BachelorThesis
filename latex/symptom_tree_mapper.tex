The aim of this component is taking the \textit{tokens for predictions} (outcome of Answer Interpreter) and with the help of the Vectorifier, output the most similar symptom's CUI.

%scrivo dello stemming
In order to do this, each symptom inside the classification must have its representative vector of the same type and dimension of the embedding of the \textit{token for prediction}. Since each symptom in the classification has a descriptive field, called ``concept name'', I treated it like a sentence and I choose to use its vectorization as representative vector. In order to hasten the whole process at runtime I pre-computed these representative vectors for each supported type and dimension.

% quanti sintomi ci sono nell'albero
\subsection{The Symptom Tree}
As mentioned in section \ref{sec:cla_symp}, a suitable data structure for representing the symptom classification is a tree. In the UMLS system this hierarchy is stored with a parent array representation. The Python library that I used for this purpose is ``anytree''.

During the construction of the tree I tagged some nodes as related with a specific body part. This is helpful during mapping: given a \textit{token for prediction} that contains a body part, the search of the correct symptom is then circumscribed only to some subtrees.

\subsection{Mapping a \textit{token for prediction} to a symptom}
This phase is divided into two parts. The first consists in passing to the Vectorifier the \textit{token for prediction} and get from it the subsentences and the relative vectors; the second in computing the cosine similarity between each subsentence vector and each symptom vector. At this point, the highest similarity shows the candidate symptom, whose CUI is predicted.

In the code there are two options that can change the process flow described above:
\begin{itemize}
  \item as mentioned earlier, if the \textit{token for prediction} contains a body part, then the mapper searches only in subtrees related to that body part (to refer to this option in the future I will use the term ``pruning'');
  \item if a \textit{minimum similarity} level is set, the prediction is output only if the highest similarity is more than this level. 
\end{itemize}
