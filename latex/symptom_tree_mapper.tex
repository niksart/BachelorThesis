The aim of this component is taking the \textit{tokens for predictions} (the outcome of the Answer Interpreter) and with the help of the Vectorifier, output the most similar symptom's CUI.

In order to do this, each symptom inside the classification must have its representative vector of the same type and dimension of the vectorified \textit{token for prediction}. Since each symptom in the classification has a descriptive field (the ``concept name''), it has been treated like a sentence and, after the stemming, its vectorization has been chosen as representative vector. In order to hasten the whole process at runtime these representative vectors were precomputed for each supported type and dimension.

\subsection{The Symptom Tree}
As mentioned in section \ref{sec:cla_symp}, a suitable data structure for representing the symptom classification is a tree. The symptom tree present in MEDCIN encompasses approximately $16\,000$ symptoms and its hierarchy is stored as a parent array representation. The Python library used for building this tree is \texttt{anytree} \cite{anytree}.

During the construction of the tree some nodes were tagged as related with a specific body part. This is helpful especially during mapping: given a \textit{token for prediction} that contains a body part, the search of the correct symptom is then circumscribed only to some subtrees.

\subsection{Mapping a \textit{token for prediction} to a symptom}
This phase is divided in two parts. The first one consists in passing to the Vectorifier the \textit{token for prediction} and get from it the sub-sentences and the relative vectors; the second one in computing the \textit{cosine similarity} between each sub-sentence vector and each symptom vector. At this point, the highest similarity shows the candidate symptom, whose CUI is predicted.

In the code there are two options that can change the algorithm described above:
\begin{itemize}
  \item as mentioned earlier, if the \textit{token for prediction} contains a body part, then the mapper searches only in subtrees related to that body part: \textit{pruning};
  \item a \textit{minimum similarity threshold}: if set, the prediction is output only if the highest similarity is more than this threshold. 
\end{itemize}
